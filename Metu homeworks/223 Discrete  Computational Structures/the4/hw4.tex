\documentclass[11pt]{article}
\usepackage[utf8]{inputenc}
\usepackage{float}
\usepackage{amsmath}
\usepackage{amssymb}

\usepackage[hmargin=3cm,vmargin=6.0cm]{geometry}
%\topmargin=0cm
\topmargin=-2cm
\addtolength{\textheight}{6.5cm}
\addtolength{\textwidth}{2.0cm}
%\setlength{\leftmargin}{-5cm}
\setlength{\oddsidemargin}{0.0cm}
\setlength{\evensidemargin}{0.0cm}

% symbol commands for the curious
\newcommand{\setZp}{\mathbb{Z}^+}
\newcommand{\setR}{\mathbb{R}}
\newcommand{\calT}{\mathcal{T}}

\begin{document}

\section*{Student Information } 
%Write your full name and id number between the colon and newline
%Put one empty space character after colon and before newline
Full Name :  Burak Bahar\\
Id Number :  2380137

% Write your answers below the section tags
\section*{Answer 1}
\paragraph{$10*(20!/18!)*(80!/72!)*6$}
\section*{Answer 2}
\paragraph{Homogeneous part\\$r^3-2r^2-15r+36=0$\\
$(r-3)^2(r+4)=0$\\
$a_n^(h) = \alpha_13^n + \alpha_2n3^n + \alpha_3(-4)^n$\\
for particular solution\\
$c*2^n$ = $c*2*2^{n-1} + c*15*2^{n-2}- c*36* 2^{n-3}+2^n$\\
$c*8*2^n$ = $8*c*2^n+30*c*2^n-36*c*2^n+8*2^n$\\
$c = 4/3$\\
$a_n^(p) = (4/3) *2^n$\\
$a_n=\alpha_13^n + \alpha_2n3^n + \alpha_3(-4)^n + (4/3) *2^n$
}
\section*{Answer 3}

\[ f=\begin{cases} 
      a_1 = 5 & $n =1$ \\
      a_n = a_{n-1}*5 & $1=n mod(2)$ \\
      a_n = a_{n-1}*10 & $0=n mod(2)$\\
   \end{cases}
\]

\section*{Answer 4}
\paragraph{\\
$\sum_{k=3}^{\infty} a_kx^{k-2} - 3 \sum_{k=2}^{\infty} a_kx^{k-1} + 3 \sum_{k=1}^{\infty} a_kx^{k} - \sum_{k=0}^{\infty} a_kx^{k+1} =0$\\
$\sum_{k=3}^{\infty} a_kx^{k} - 3 \sum_{k=2}^{\infty} a_kx^{k+1} + 3 \sum_{k=1}^{\infty} a_kx^{k+2} - \sum_{k=0}^{\infty} a_kx^{k+3} =0$\\
let $f(x) = \sum_{k=0}^{\infty} a_kx^{k}$ then \\
$(f(x)-a_2-a_1-a_0)-3x(f(x)-a_1-a_0)+3x^2(f(x)-a_0)-x^3f(x)=0$\\
$f(x)=(3x^2-12x+10)/-x^3+3x^2-3x+1=(3x^2-12x+10)/(1-x)^3$\\
$A/(1-x)+ B/(1-x)^2 + C/(1-x)^3$\\
$Ax^2-x(2A+B)+A+B+C=3x^2-12x+10$\\
$A=3$ , $B=6$ , $C=1$\\
$f(x)=3/(1-x)+ 6/(1-x)^2 + 1/(1-x)^3 $\\
from table (rows = 5,8,9)\\
$a_k=3*1^k + 6(k+1)+ C(3+k-1,k)$}
\section*{Answer 5}
\paragraph{a.\\
reflexive \\
for $((a,b),(b,a)) \in R$  \\  $ a+b = b+a$\\
transitive\\
for $((a,b),(c,d)) \in R$  \\  $ a+d = b+c$\\
for $((b,e),(d,f)) \in R$\\
$b+f = e+d$\\
for $((a,e),(c,f)) \in R$\\
$a+f=e+c$\\
symmetric\\
for $((d,c),(b,a)) \in R$ \\ $b+c = a+d$
}
\paragraph{b.\\
for $(c,d) = $\{ $d-c =1$ $| $ $ (c,d) \in R$ \}}


\end{document}